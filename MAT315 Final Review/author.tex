%%%%%%%%%%%%%%%%%%%% author.tex %%%%%%%%%%%%%%%%%%%%%%%%%%%%%%%%%%%
%
% sample root file for your "contribution" to a contributed volume
%
% Use this file as a template for your own input.
%
%%%%%%%%%%%%%%%% Springer %%%%%%%%%%%%%%%%%%%%%%%%%%%%%%%%%%


% RECOMMENDED %%%%%%%%%%%%%%%%%%%%%%%%%%%%%%%%%%%%%%%%%%%%%%%%%%%
\documentclass[graybox]{svmult}

% choose options for [] as required from the list
% in the Reference Guide

\usepackage{mathptmx}       % selects Times Roman as basic font
\usepackage{helvet}         % selects Helvetica as sans-serif font
\usepackage{courier}        % selects Courier as typewriter font
\usepackage{type1cm}        % activate if the above 3 fonts are
                            % not available on your system
%
\usepackage{makeidx}         % allows index generation
\usepackage{graphicx}        % standard LaTeX graphics tool
                             % when including figure files
\usepackage{multicol}        % used for the two-column index
\usepackage[bottom]{footmisc}% places footnotes at page bottom
\usepackage{hyperref}
\usepackage{amsmath}
\usepackage{amssymb}
\usepackage{enumerate}

% see the list of further useful packages
% in the Reference Guide

\makeindex             % used for the subject index
                       % please use the style svind.ist with
                       % your makeindex program

%%%%%%%%%%%%%%%%%%%%%%%%%%%%%%%%%%%%%%%%%%%%%%%%%%%%%%%%%%%%%%%%%%%%%%%%%%%%%%%%%%%%%%%%%

\begin{document}

\title*{MAT315: Intro to Number Theory Final Review}
% Use \titlerunning{Short Title} for an abbreviated version of
% your contribution title if the original one is too long
\author{Rui Qiu}

\maketitle

The Instructor this semester was \href{http://www.math.toronto.edu/henrykim/henrykim.html}{Henry Kim}.
The following material will mostly cover the important definitions, theorems, algorithms and/or (partial) proofs taught in class.

Note: The shaded theorems would be extremenly valuable in the final test.

\section*{Chapter 2: Pythagorean Triples}

\begin{definition}
A \textit{primitive Pythagorean triple} (or PPT for short) is a triple of numbers ($a, b, c$) such that $a, b,$ and $c$ have no common factors and satisfy
\begin{align*} 
a^2 + b^2 = c^2.
\end{align*}
\end{definition}

\begin{svgraybox}
\begin{theorem}
(Pythagorean Triples Theorem). We will get every primitive Pythagorean triple (a, b, c) 
with a odd and b even by using the formulas
\begin{align*}
a = st, \\
b = \frac{s^2-t^2}{2}, \\
c = \frac{s^2+t^2}{2},
\end{align*}
where $s > t \geqslant 1$ are chosen to be any odd integers with no common factors.
\end{theorem}
\end{svgraybox}

\section*{Chapter 3: Pythagorean Triples and the Unit Circle}

\begin{theorem}
Every point on the circle
\begin{align*}
x^2 + y^2 = 1
\end{align*}
whose coordinates are rational numbers can be obtained from the formula
\begin{align*}
(x, y) = \left(\frac{1-m^2}{1+m^2}, \frac{2m}{1+m^2}\right)
\end{align*}
by substituting in rational numbers for m $\lbrack$except for the point $(-1, 0)$ which is the limiting value as m$\rightarrow\infty \rbrack$.
\end{theorem}

Note: The process of getting this formula involves solving a quadratic equation. The trick is to plug in "known" solution term.

If we write the rational number $m$ as a fraction $\frac{v}{u}$, then our formula becomes
\begin{align*}
(x,y) = \left(\frac{u^2-v^2}{u^2+v^2}, \frac{2uv}{u^2+v^2}\right),
\end{align*}
and clearing denominators gives the Pythagorean triple
\begin{align*}
(a,b,c)=(u^2-v^2,2uv,u^2+v^2).
\end{align*}
Note that if we use symbol $s, t$ in Chapter 2, we can set
\begin{align*}
u = \frac{s+t}{2} \\
v = \frac{s-t}{2}
\end{align*}

\section*{Chapter 5: Divisibility and the Greatest Common Divisor}

\begin{definition}
The \textit{greatest common divisor} of two numbers $a$ and $b$ (not both zero) is the largest number that divides both of them. It is denoted by $gcd(a,b)$. If $gcd(a,b)=1$, we say that $a$ and $b$ are \textit{relatively prime}.
\end{definition}

\begin{svgraybox}
\begin{theorem}
(Euclidean Algorithm). To compute the greatest common divisor of two number a and b, let $r_{-1} = a$, let $r_0 = b$, and compute successive quotients and remainders
\begin{align*}
r_{i-1} = q_{i+1} \times r_i + r_{i+1}
\end{align*}
for $i = 0, 1, 2,...$ until some remainder $r_{n+1}$ is 0. The last nonzero remainder $r_n$ is then the greatest common divisor of a and b.
\end{theorem}
\end{svgraybox}

\section*{Chapter 6: Linear Equations and the Greatest Common Divisor}

The smalleste positive value of $ax + by$ is equal to $gcd(a, b)$.

\begin{theorem}
(Linear Equation Theorem). Let a and b be nonzero integers, and let $g=gcd(a,b)$. The equation
\begin{align*}
ax+by=g
\end{align*}
always has a solution $(x_1, y_1)$ in integers, and this solution can be found by the Euclidean algorithm method described earlier. The every solution to the equation can be obtained by substituting integers k into the formula
\begin{align*}
\left(x_1+k\cdot\frac{b}{g}, y_1-k\cdot\frac{a}{g}\right).
\end{align*}
\end{theorem}

\section*{Chapter 7: Factorization and the Fundamental Theorem of Arithmetic}

\begin{definition}
A \textit{prime number} is a number $p \geqslant 2$ whose only (positive) divisors are 1 and $p$. Numbers $m \geqslant 2$ that are not primes are called \textit{composite numbers}.
\end{definition}

\begin{lemma}
Let $p$ be a prime number, and suppose that $p$ divides the product $ab$. Then either $p$ divides $a$ or $p$ divides $b$ (or $p$ divides both $a$ and $b$).
\end{lemma}

\begin{theorem}
(Prime Divisibility Property). Let $p$ be a prime number, and suppose that $p$ divides the product $a_1a_2\cdots a_r$. Then $p$ divides at least one of the factors $a_1, a_2, \ldots, a_r$.
\end{theorem}

\begin{theorem}
(The Fundamental Theorem of Arithmetic). Every integer $n \geqslant 2$ can be factored into a product of primes
\begin{align*}
n=p_1p_2\cdots p_r
\end{align*}
in exactly one way.
\end{theorem}

\section*{Chapter 8: Congruences}

\begin{definition}
We say that $a$ is \textit{congruent to} $b$ \textit{modulo} $m$, and we write $a \equiv b\bmod m$, if $m$ divides $a-b$.
\end{definition}

\begin{theorem}
(Linear Congruence Theorem). Let $a,c$ and $m$ be integers with $m\geqslant 1$, and let $g=gcd(a,m)$.
\begin{enumerate}[(a)]
\item If $g \nmid c,$ then the congruence $ax \equiv c \bmod m$ has no solutions.
\item If $g \mid c$, then the congruence $ax \equiv c \bmod m$ has exactly $g$ incongruent solutions. To find the solutions, first find a solution $(u_0, v_0)$ to the linear equation
\begin{align*}
au+mv=g.
\end{align*}
(A method for solving this equation is described in Chapter 6.) Then $x_0 = \frac{cu_0}{g}$ is a solution to $ax \equiv c \bmod m$, and a complete set of incongruent solutions is given by
\begin{align*}
x \equiv x_0 + k \cdot \frac{m}{g} \pmod m \text{ for } k = 0,1,2,\ldots,g-1.
\end{align*}
\end{enumerate}
\end{theorem}

\begin{theorem}
(Polynomial Roots Mod $p$ Theorem). Let $p$ be a prime number and let
\begin{align*}
f(x) = a_0x^d+a_1x^{d-1}+\cdots+a_d
\end{align*}
be a polynomial of degree $d \geqslant 1$ with integer coefficients and with $p \nmid a_0$. Then the congruence
\begin{align*}
f(x) \equiv 0 \pmod p
\end{align*}
has at most $d$ incongruent solutions.
\end{theorem}

\section*{Chapter 9: Congruences, Powers, and Fermat's Little Theorem}

\begin{svgraybox}
\begin{theorem}
(Fermat's Little Theorem). Let $p$ be a prime number, and let $a$ be any number with $a \not \equiv 0 \pmod p$. Then
\begin{align*}
a^{p-1} \equiv 1 \pmod p.
\end{align*}
\end{theorem}
\end{svgraybox}

\begin{lemma}
Let $p$ be a prime number and let $a$ be a number with $a \not \equiv 0 \pmod p$. Then the numbers
\begin{align*}
a, 2a, 3a, \ldots, (p-1)a \pmod p
\end{align*}
are the same as the numbers
\begin{align*}
1, 2, 3, \ldots, (p-1) \pmod p,
\end{align*}
although they may be in a different order.
\end{lemma}

\section*{Chapter 10: Congruences, Powers, and Euler's Formula}

\begin{definition}
The number of integers between $1$ and $m$ that are relatively prime to $m$ is an important quantity, so we give this quantity a name:
\begin{align*}
\phi(m) = \#\{a : 1 \leqslant a \leqslant m \text{ and } gcd(a,m) = 1\}.
\end{align*}
The function $\phi$ is called \textit{Euler's phi function}.
\end{definition}

\begin{theorem}
(Euler's Formula). If $gcd(a, m)=1$, then
\begin{align*}
a^{\phi(m)} \equiv 1 \pmod m.
\end{align*}
\end{theorem}

\begin{lemma}
If $gcd(a,m) = 1$, then the numbers
\begin{align*}
b_1a,b_2a,b_3a,\ldots,b_{\phi(m)}a \pmod m
\end{align*}
is congruent to one number in the list
\begin{align*}
b_1,b_2,b_3,\ldots,b_{\phi(m)} \pmod m.
\end{align*}
\end{lemma}

\section*{Chapter 11: Euler's Phi Function and the Chinese Remainder Theorem}

\begin{theorem}
(Phi Function Formulas).
\begin{enumerate}[(a)]
\item If $p$ is a prime and $k \geqslant 1$, then
\begin{align*}
\phi(p^k)=p^k - p^{k-1}.
\end{align*}
\item If $gcd(m, n)=1$, then $\phi(mn) = \phi(m)\phi(n)$.
\end{enumerate}
\end{theorem}

\begin{svgraybox}
\begin{theorem}
(Chinese Remainder Theorem). Let $m$ and $n$ be integers satisfying $gcd(m,n)=1$, and let $b$ and $c$ be any integers. Then the simultaneous congruences
\begin{align*}
x \equiv b \pmod m \text{ and } x \equiv c \pmod n.
\end{align*}
have exactly one solution with $0 \leqslant x < mn$.
\end{theorem}
\end{svgraybox}

Note: There's always a general solution for CRT. How to solve? Substitution + Euclidean Algorithm.

\section*{Chapter 12: Prime Numbers}

\begin{theorem}
(Infinitely Many Prime Theorem). There are infinitely many prime numbers.
\end{theorem}

\textit{Euclid's Proof.} Suppose we have some list of primes $p_1,p_2,\ldots,p_r$. we multiply them together and add 1, which gives the number
\begin{align*}
A = p_1p_2\cdots p_r +1.
\end{align*}
If $A$ itself a prime, we're done, since $A$ is too large to be in the original list. But even if $A$ is not prime, it will certainly be divisible by some prime, since every number can be written as a product of primes. Let $q$ be some prime dividing $A$, for example, the smallest one. I claim that $q$ is not in the original list, so it will be the desired new prime.

Why isn't $q$ in the original list? We know $q$ divides $A$, so
\begin{align*}
q \text{ divides } p_1p_2\ldots p_r +1.
\end{align*}
If $q$ were to equal one of the $p_i$'s, then it would have to divide 1, which is not possible. This means $q$ is a new prime that may be added to our list. Repeating this process, we can create a list of primes that is as long as we want. This shows that there must be infinitely many prime numbers.

\begin{theorem}
(Prime 3 (Mod 4) Theorem). There are infinitely many primes that are congruent to 3 modulo 4.
\end{theorem}

\textit{Proof:} We suppose that we have already compiled a (finite) list of primes, all of which are congruent to 3 modulo 4. Our goal is to make the list longer by finding a new 3 modulo 4 prime. Repeating this process gives a list of any desired length, thereby proving that there are infinitely many primes congruent to 3 modulo 4.

Suppose that our initial list of primes congruent to 3 modulo 4 is 
\begin{align*}
3,p_1,p_2,\ldots,p_r.
\end{align*}
Consider the number
\begin{align*}
A = 4p_1p_2\cdots p_r +3.
\end{align*}
(Notice that we don't include the prime 3 in the product.) We know that $A$ can be factored into a product of primes, say
\begin{align*}
A = q_1q_2\cdots q_s.
\end{align*}
I claim that among the primes $q_1, q_2, \ldots, q_s$ at least one of them must be congruent to 3 modulo 4. This is the key step in the proof. Why is it true? If not, then $q_1, q_2, \ldots, q_s$ would all be congruent to 1 modulo 4, in which case their product $A$ would be congruent to 1 modulo 4. But you can see from its definition that $A$ is clearly congruent to 3 modulo 4. Hence, at least one of $q_1, q_2, \ldots, q_s$ must be congruent to 3 modulo 4, say $q_i \equiv 3 \bmod 4$.

My second claim is that $q_i$ is not in the original list. Why not? We know that $q_i$ divides $A$, while it is clear from the definition of $A$ that none of $3, p_1, p_2, \ldots, p_r$ divides $A$. Thus, $q_i$ is not in our original list, so we may add it to the list and repeat process. In this way we can create as long a list as we want, which shows that there must be infinitely many primes congruent to 3 modulo 4.

\begin{theorem}
(Dirichlet's Theorem on Primes in Arithmetic Progressions). Let $a$ and $m$ be integers with $gcd(a, m)= 1$. Then there are infinitely many primes that are congruent $a$ modulo $m$. That is, there are infinitely many prime numbers $p$ satisfying
\begin{align*}
p \equiv a \bmod m.
\end{align*}
\end{theorem}

\section*{Chapter 14: Mersenne Primes}

\begin{definition}
Primes of the form $2^p - 1$ are called \textit{Mersenne primes}
\end{definition}

Note: Not every $2^p-1$ is prime.

\section*{Chapter 15: Mersenne Primes and Perfect Numbers}

\begin{definition}
A \textit{perfect number} is a number that is equal to the sum of its proper divisors. The proper divisors of a number are the divisors other than itself.
\end{definition}

\begin{theorem}
(Euclid's Perfect Number Formula). If $2^p - 1$ is a prime number, then $2^{p-1}(2^p-1)$ is a perfect number.
\end{theorem}

\begin{theorem}
(Euler's Perfect Number Theorem). If $n$ is an even perfect number, then $n$ looks like
\begin{align*}
n = 2^{p-1}(2^p-1),
\end{align*}
where $2^p-1$ is a Mersenne prime.
\end{theorem}

\begin{definition}
A \textit{sigma function} is equal to $\sigma(n) = \text{ sum of all divisors of } n \\
\text{(including 1 and } n \text{ )}$. 
\end{definition}

\begin{theorem}
(Sigma Function Formulas).
\begin{enumerate}[(a)]
\item If $p$ is a prime and $k \geqslant 1$, then
\begin{align*}
\sigma(p^k) = 1 + p + p^2 + \cdots + p^k = \frac{p^{k+1}-1}{p-1}.
\end{align*}
\item If $gcd(m, n) = 1$, then
\begin{align*}
\sigma(mn) = \sigma(m)\sigma(n).
\end{align*}
\end{enumerate}
\end{theorem}

A number $n$ is perfect if the sum of its divisors, other than $n$ itself, is equal to $n$. The sigma function $\sigma(n)$ is the sum of the divisors of $n$, including $n$, so it has an extra $n$. Therefore,
\begin{align*}
n \text{ is perfect exactly when } \sigma(n) = 2n.
\end{align*}

\section*{Chapter 16: Powers Modulo $m$ and Successive Squaring}

\textbf{Algorithm }(Successive Squaring to Compute $a^k \bmod m$). The following steps compute the value of $a^k \bmod m$:
\begin{enumerate}
\item Write $k$ as a sum of powers of 2,
\begin{align*}
k = u_0 + u_1 \cdot 2 + u_2 \cdot 4 + u_3 \cdot 8 + \cdots + u_r \cdot 2^r,
\end{align*}
where each $u_i$ is either 0 or 1. (This is called \textit{the binary expansion of} $k$.)
\item Make a table of powers of $a$ modulo $m$ using successive squaring.
\begin{align*}
a^1 \equiv A_0 \bmod m\\
a^2 \equiv (a^1)^2 \equiv A_0^2 \equiv A_1 \bmod m\\
a^4 \equiv (a^2)^2 \equiv A_1^2 \equiv A_2 \bmod m\\
a^8 \equiv (a^4)^2 \equiv A_2^2 \equiv A_3 \bmod m\\
\cdots\\
a^{2r} \equiv \left(a^{2^{r-1}}\right)^2\equiv A_{r-1}^2 \equiv A_r \bmod m
\end{align*}
Note that to compute each line of the table you only need to take the number at the end of the previous line, square it, and then reduce it modulo $m$. Also note that the table has $r+1$ lines, where $r$ is the highest exponent of 2 appearing in the binary expansion of $k$ in Step 1.
\item The product
\begin{align*}
A_0^{u_0}\cdot A_1^{u_1} \cdot A_2^{u_2} \cdots A_r^{u_r} \bmod m
\end{align*}
will be congruent to $a^k \pmod m$. Note that all the $u_i$'s are either 0 or 1, so this number is really the product of those $A_i$'s for which $u_i$ equals 1.
\end{enumerate}

Using successive squaring and Fermat's Little Theorem, we can show that a number $m$ is composite without finding any factors. Take any number $a$ less than $m$. First compute $gcd(a,m)$. If it is greater than 1, then it's a factor of $m$, we are done. If not, if $gcd(a,m) = 1$, use successive squaring to compute
\begin{align*}
a^{m-1} \bmod m.
\end{align*}
Fermat's Little Theorem says that if $m$ is prime then the answer will be 1.

But numbers like \textit{Carmichael numbers} do exist, and those composite numbers $m$ do satisfy the equation $a^{m-1} \equiv 1 \bmod m$ for all $a$'s with $gcd(a,m)=1$. The smallest \textit{Carmichael number} is 561.

\section*{Chapter 17: Computing $k^{th}$ Roots Modulo $m$}

\textbf{Algorithm }(How to Compute $k^{th}$ Roots Modulo $m$). Let $b, k,$ and $m$ be given integers that satisfy
\begin{align*}
gcd(b,m)=1 \text{ and } gcd(k, \phi(m))=1.
\end{align*}
The following steps give a solution to the congruence
\begin{align*}
x^k \equiv b \bmod m.
\end{align*}
\begin{enumerate}
\item Compute $\phi(m)$.
\item Find positive integers $u$ and $v$ that satisfy $ku-\phi(m)v=1$.
\item Compute $b^u \bmod m$ by successive squaring. The value obtained gives the solution $x$.
\end{enumerate}

\section*{Chapter 20: Squares Modulo $p$}

\begin{definition}
A nonzero number that is congruent to a square module $p$ is called a \textit{quadratic residue modulo p (QR)}. A number that is not congruent to a square modulo $p$ is called a \textit{(quadratic) nonresidue modulo p (NR)}.
\end{definition}

\begin{theorem}
Let $p$ be an odd prime. Then there are exactly $\frac{p-1}{2}$ quadratic residues modulo p and exactly $\frac{p-1}{2}$ nonresidues modulo p.
\end{theorem}

\begin{theorem}
(Quadratic Residue Multiplication Rule). (Version 1) Let $p$ be an odd prime. Then:
\begin{enumerate}[(i)]
\item QR $\times$ QR = QR,
\item QR $\times$ NR = NR,
\item NR $\times$ NR = QR.
\end{enumerate}
\end{theorem}

QR behaves like +1 and NR behaves like -1.

\begin{definition}
The \textit{Legendre symbol} of $a$ modulo $p$ is
\begin{align*}
\left(\frac{a}{p}\right) = \left\{
\begin{array}{c l}      
    1 & \text{if $a$ is a quadratic residue modulo $p$,}\\
    -1 & \text{if $a$ is a nonresidue modulo $p$.}
\end{array}\right.
\end{align*}
\end{definition}

\begin{theorem}
(Quadratic Residue Multiplication Rule). (Version 2) Let $p$ be an odd prime. Then
\begin{align*}
\left(\frac{a}{p}\right)\left(\frac{b}{p}\right)=\left(\frac{ab}{p}\right).
\end{align*}
\end{theorem}

\section*{Chapter 21: Is $-1$ a Square Modulo $p$? Is 2?}

\begin{theorem}
(Euler's Criterion). Let $p$ be an odd prime. Then
\begin{align*}
a^{\frac{p-1}{2}} \equiv \left(\frac{a}{p}\right) \bmod p.
\end{align*}
\end{theorem}

\begin{svgraybox}
\begin{theorem}
(Quadratic Reciprocity). (Part I) Let $p$ be an odd prime. Then
\begin{align*}
-1 \text{ is a quadratic residue modulo } p \text{ if } p \equiv 1 \bmod 4 \text{, and} \\
-1 \text{ is a nonresidue modulo } p \text{ if } p \equiv 3 \bmod 4.
\end{align*}
In other words, using the Legendre symbol,
\begin{align*}
\left(\frac{-1}{p}\right) = \left\{
\begin{array}{c l}
    1 & \text{if } p \equiv 1 \bmod 4, \\
    -1 & \text{if } p \equiv 3 \bmod 4.
\end{array}\right.
\end{align*}
\end{theorem}
\end{svgraybox}

\begin{theorem}
(Primes 1 (Mod 4) Theorem). There are infinitely many primes that are congruent to 1 modulo 4.
\end{theorem}

\textit{Proof.} Suppose given a list of primes $p_1, p_2, \ldots, p_r$, all of which are congruent to 1 modulo 4. Consider the number
\begin{align*}
A = (2p_1p_2\cdots p_r)^2 +1.
\end{align*}
We know that $A$ can be factored into a product of primes, say
\begin{align*}
A = q_1q_2\cdots q_s.
\end{align*}
It's clear that $q_1, q_2, \ldots, q_s$ are not in our original list, since none of the $p_i$'s divide $A$. So all we need to do is show that one of the $q_i$'s is congruent to 1 modulo 4. In fact, we'll see all of them are.

First note that $A$ is odd, so all the $q_i$'s are odd. Next, each $q_i$ divides $A$, so
\begin{align*}
(2p_1p_2\cdots p_r)^2 + 1 = A \equiv 0 \bmod q_i.
\end{align*}
This means that $x = 2p_1p_2\cdots p_r$ is a solution to the congruence
\begin{align*}
x^2 \equiv -1 \bmod q_i,
\end{align*}
so -1 is a quadratic residue modulo $q_i$. Now Quadratic Reciprocity tells us that $q_i \equiv 1 \bmod 4$.

\begin{svgraybox}
\begin{theorem}
(Quadratic Reciprocity). (Part II). Let $p$ be an odd prime. Then 2 is a quadratic residue modulo $p$ if $p$ is congruent to 1 or 7 modulo 8, and 2 is a nonresidue modulo $p$ if $p$ is congruent to 3 or 5 modulo 8. In terms of the Legendre symbol,
\begin{align*}
\left(\frac{2}{p}\right) = \left\{
\begin{array}{c l}
    1 & \text{if } p \equiv 1 \text{ or } 7 \bmod 8, \\
    -1 & \text{if } p \equiv 3 \text{ or } 5 \bmod 8.
\end{array}\right.
\end{align*}
\end{theorem}
\end{svgraybox}

\section*{Chapter 22: Quadratic Reciprocity}

\begin{svgraybox}
\begin{theorem}
(Law of Quadratic Reciprocity). Let $p$ and $q$ be distinct odd primes.
\begin{align*}
\left(\frac{-1}{p}\right) = \left\{
\begin{array}{c l}
    1 & \text{if } p \equiv 1 \bmod 4, \\
    -1 & \text{if } p \equiv 3 \bmod 4.
\end{array}\right. \\
\left(\frac{2}{p}\right) = \left\{
\begin{array}{c l}
    1 & \text{if } p \equiv 1 \text{ or } 7 \bmod 8, \\
    -1 & \text{if } p \equiv 3 \text{ or } 5 \bmod 8.
\end{array}\right. \\
\left(\frac{q}{p}\right) = \left\{
\begin{array}{c l}
    \left(\frac{p}{q}\right) & \text{if } p \equiv 1 \bmod 4 \text{ or } q \equiv 1 \bmod 4, \\
    -\left(\frac{p}{q}\right) & \text{if } p \equiv 3 \bmod 4 \text{ and } q \equiv 3 \bmod 4.
\end{array}\right.
\end{align*}
\end{theorem}
\end{svgraybox}

\begin{svgraybox}
\begin{theorem}
(Generalized Law of Quadratic Reciprocity). Let $a$ and $b$ be odd positive integers.
\begin{align*}
\left(\frac{-1}{b}\right) = \left\{
\begin{array}{c l}
    1 & \text{if } b \equiv 1 \bmod 4, \\
    -1 & \text{if } b \equiv 3 \bmod 4.
\end{array}\right. \\
\left(\frac{2}{b}\right) = \left\{
\begin{array}{c l}
    1 & \text{if } b \equiv 1 \text{ or } 7 \bmod 8, \\
    -1 & \text{if } b \equiv 3 \text{ or } 5 \bmod 8.
\end{array}\right. \\
\left(\frac{a}{b}\right) = \left\{
\begin{array}{c l}
    \left(\frac{a}{b}\right) & \text{if } a \equiv 1 \bmod 4 \text{ or } b \equiv 1 \bmod 4, \\
    -\left(\frac{a}{b}\right) & \text{if } a \equiv b \equiv 3 \bmod 4.
\end{array}\right.
\end{align*}
\end{theorem}
\end{svgraybox}

\section*{Chapter 23: Proof of Quadratic Reciprocity}

\begin{definition}
Consider a list of numbers $a, 2a, 3a, \ldots, Pa$, and we reduce them
modulo $p$ into the range from $-P$ to $P$. some of the reduced values will be positive
and some of them will be  negative.

Let $\mu(a, p) =$ number of integers in the list that become negative when the integers in the list are reduced to modulo $p$ into the interval from $-P$ to $P$.
\end{definition}

\begin{theorem}
(Gauss's Criterion). Let $p$ be an odd prime, let $a$ be an integer that is not divisible by $p$, and let $\mu(a,p)$ be the number defined previously. Then
\begin{align*}
\left(\frac{a}{p}\right) = (-1)^{\mu(a, p)}.
\end{align*}
\end{theorem}

\begin{lemma}
When the numbers $a, 2a, 3a, \ldots, Pa$ are reduced modulo $p$ into the range from $-P$ to $P$, the reduced values are $\pm1,\ldots,\pm P$ in some order, with each number appearing once with either a plus sign or a minus sign.
\end{lemma}

\begin{lemma}
Let $p$ be an odd prime, let $P=\frac{p-1}{2}$, let $a$ be an odd integer that is not divisible by $p$, and let $\mu(a, p)$ be the quantity defined previously that appears in Gauss's criterion. Then
\begin{align*}
\sum_{k=1}^{P} \lfloor \frac{ka}{p} \rfloor \equiv \mu(a,p) \bmod 2.
\end{align*}
\end{lemma}

\begin{definition}
\textit{Jacobi symbol}: $n$ odd positive integer, $gcd(a,n)=1$, $n = p_1\cdots p_r$ product of prime, $p_i$'s are not necessarily distinct. Define
\begin{align*}
\left(\frac{a}{n}\right) = \left(\frac{a}{p_1}\right)\cdots\left(\frac{a}{p_r}\right).
\end{align*}
With following properties:
\begin{enumerate}
\item If $a \equiv a' \bmod n, \left(\frac{a}{n}\right)=\left(\frac{a'}{n}\right).$
\item $\left(\frac{ab}{n}\right)=\left(\frac{a}{n}\right)\left(\frac{b}{n}\right).$
\item $\left(\frac{a}{mn}\right)=\left(\frac{a}{m}\right)\left(\frac{a}{n}\right).$
\item If $x^2\equiv a \bmod n$ has a solution, then $\left(\frac{a}{n}\right) = 1.$
\end{enumerate}
\end{definition}

\section*{Chapter 24: Which Primes Are Sums of Two Squares?}

\begin{theorem}
(Sum of Two Squares Theorem for Primes). Let $p$ be a prime. Then $p$ is a sum of two squares exactly when
\begin{align*}
p \equiv 1 \bmod 4 \text{ or $p=2$}.
\end{align*}
\end{theorem}

Know that $A^2 +B^2 = Mp$ for some integers $A, B,$ and $M$. What to find integers $a, b,$ and $m$ with $a^2 +b^2 = mp$ and $m\leqslant M-1$.

Denote the identity that $(u^2 + v^2)(A^2+B^2) = (uA+vB)^2 + (vA-uB)^2.$

\textbf{Descent Procedure}
\begin{enumerate}
\item $p$ any prime $\equiv 1 \bmod 4$
\item Write $A^2 +B^2 = Mp$ with $M < p$
\item Choose numbers $u$ and $v$ with $u\equiv A \bmod M, v\equiv B \bmod M, -\frac{1}{2}M \leqslant u, v \leqslant \frac{1}{2}M$
\item Observe that $u^2 + v^2 \equiv A^2 + B^2 \equiv 0 \bmod M$
\item So we can write $u^2 + v^2 = Mr, A^2 +B^2 \equiv Mp$ (for some $1 \leqslant r < M$)
\item Multiply to get $(u^2 + v^2)(A^2+B^2)=M^2rp$
\item Use the identity above.
\item $(uA+vB)^2 +(vA-uB)^2 = M^2rp$
\item Divide by $M^2$. $\left(\frac{uA+vB}{M}\right)^2 + \left(\frac{vA-uB}{M}\right)^2 = rp$
\item Repeat this process until $p$ itself is written as a sum of two squares
\end{enumerate}

\section*{Chapter 25: Which Numbers Are Sums of Two Squares?}

\textbf{Divide and Conquer:}
Divide: Factor $m$ into a product of primes $p_1p_2\cdots p_r$.
Conquer: Write each prime $p_i$ as a sum of two squares.
Unify: Use the identity $(u^2+v^2)(A^2+B^2)=(uA+vB)^2+(vA-uB)^2$ repeatedly to write $m$ as a sum of two squares.

\begin{svgraybox}
\begin{theorem}
(Sum of Two Squares Theorem). Let $m$ be a positive integer.
\begin{enumerate}[(a)]
\item Factor $m$ as
\begin{align*}
m=p_1p_2\cdots p_rM^2
\end{align*}
with distinct prime factors $p_1, p_2, \ldots, p_r$. Then $m$ can be written as a sum of two squares exactly when every $p_i$ is either 2 or is congruent to 1 modulo 4.
\item The number $m$ can be written as a sum of two squares $m=a^2+b^2$ with $gcd(a,b)=1$ if and only if it satisfies one of the following two conditions:
\begin{enumerate}[(i)]
\item $m$ is odd and every prime divisor of $m$ is congruent to 1 modulo 4.
\item $m$ is even, $\frac{m}{2}$ is odd, and every prime divisor of $\frac{m}{2}$ is congruent to 1 modulo 4.
\end{enumerate}
\end{enumerate}
\end{theorem}
\end{svgraybox}

\begin{theorem}
(Pythagorean Hypotenuse Proposition). A number $c$ appears as the hypotenuse of a primitive Pythagorean triple $(a,b,c)$ if and only if $c$ is a product of primes each of which is congruent to 1 modulo 4.
\end{theorem}

\section*{Chapter 27: Euler's Phi Function and Sums of Divisors}

Recall Euler's phi functions for primes: $\phi(p^k) = p^k -p^{k-1}$.
Define a function $F(n)$ by the formula: $F(n)=\phi(d_1)+\phi(d_2)+\cdots + \phi(d_r)$, where $d_1, d_2, \ldots , d_r$ are the divisors of $n$.

\begin{lemma}
If $gcd(m,n)=1$, then $F(mn)=F(m)F(n)$.
\end{lemma}

\begin{theorem}
(Euler's Phi Function Summation Formula). Let $d_1, d_2, \ldots, d_r$ be the divisors of $n$. Then
\begin{align*}
\phi(d_1)+\phi(d_2)+\cdots +\phi(d_r)=n
\end{align*}
\end{theorem}

\section*{Chapter 28: Powers Modulo $p$ and Primitive Roots}

\begin{definition}
The \textit{order of a modulo p} is $e_p(a)=$ the smallest exponent $e \geqslant 1$ such that $a^e\equiv 1 \bmod p$).
\end{definition}

\begin{theorem}
(Order Divisibility Property). Let $a$ be an integer not divisible by the prime $p$, and suppose that $a^n \equiv 1 \bmod p$. Then the order $e_p(a)$ divides $n$.
In particular, the order $e_p(a)$ always divides $p-1$.
\end{theorem}

\begin{definition}
A number $g$ with maximum order $e_p(g) = p-1$ is called a \textit{primitive root modulo p}.
\end{definition}

For example, $p=7, 1^1 \equiv 1 \bmod 7, 2^3 \equiv 1 \bmod 7, 3^6 \equiv 1 \bmod 7, 4^3 \equiv 1 \bmod 7, 5^6\equiv 1 \bmod 7, 6^2 \equiv 1 \bmod 7$. So the primitive roots modulo 7 are 3 and 5.

\begin{svgraybox}
\begin{theorem}
(Primitive Root Theorem). Every prime $p$ has a primitive root. More precisely, there are exactly $\phi(p-1)$ primitive roots modulo p.
\end{theorem}
\end{svgraybox}

\section*{Chapter 29: Primitive Roots and Indices}

\begin{definition}
For any number $1 \leqslant a < p$, we can pick out exactly one of the powers $g, g^2, g^3, \ldots, g^{p-3}, g^{p-2}, g^{p-1}$ as being congruent to $a$ modulo $p$. The exponent is called the \textit{index of a modulo p for the base g}. Write $I(a)$ for the index.
\end{definition}

If we use the primitive root 2 as base for the prime 13, then $I(3) = 4$, since $2^4 =16\equiv 3 \bmod 13$. 

\begin{theorem}
(Rules for Indices). Indices satisfy the following rules:
\begin{enumerate}[(a)]
\item $I(ab)\equiv I(a) + I(b) \bmod (p-1)$ [Product Rule]
\item $I(a^k)\equiv kI(a) \bmod (p-1)$ [Power Rule]
\end{enumerate}
\end{theorem}

\section*{Chapter 31: Square-Triangular Numbers Revisited}

\begin{theorem}
(Square-Triangular Number Theorem).
\begin{enumerate}[(a)]
\item Every solution in positive integers to the equation
\begin{align*}
x^2-2y^2=1
\end{align*}
is obtained by raising $3+2\sqrt2$ to powers. That is, the solutions $(x_k,y_k)$ can all be found by multiplying out
\begin{align*}
x_k+y_k\sqrt2=(3+2\sqrt2)^k \text{ for } k=1, 2, 3, \ldots
\end{align*}
\item Every square-triangular number $n^2=\frac{1}{2}m(m+1)$ is given by
\begin{align*}
m = \frac{x_k-1}{2}, n = \frac{y_k}{2} \text{ for } k=1,2,3,\ldots,
\end{align*}
where the $(x_k, y_k)$'s are the solutions from (a).
\end{enumerate}
\end{theorem}

\section*{Chapter 32: Pell's Equation}

\begin{definition}
A \textit{Pell's equation} is an equation of the form $x^-Dy^2=1$ where $D$ is a fixed positive integer that is not a perfect square.
\end{definition}

\begin{svgraybox}
\begin{theorem}
(Pell's Equation Theorem). Let $D$ be a positive integer that is not a perfect square. Then Pell's equation
\begin{align*}
x^2-Dy^2=1
\end{align*}
always has solutions in positive integers. If $(x_1, y_1)$ is the solution with smallest $x_1$, then every solution $(x_k, y_k)$ can be obtained by taking powers
\begin{align*}
x_k+y_k\sqrt{D}=(x_1+y_1\sqrt{D})^k \text{ for } k = 1,2,3,\ldots.
\end{align*}
\end{theorem}
\end{svgraybox}

There is no known pattern as to when the smallest solution is actually small and when it is large.

\section*{Chapter 35: Number Theory and Imaginary Numbers}

\begin{theorem}
(The Fundamental Theorem of Algebra). If $a_0,a_1,a_2,\ldots,a_d$ are complex numbers with $a_0 \not = 0$ and $d \geqslant1$, then the equation
\begin{align*}
a_0x^d+a_1x^{d-1}+a_2x^{d-2} +\cdots +a_{d-1}x+a_d=0
\end{align*}
has a solution in complex numbers.
\end{theorem}

\begin{definition}
The \textit{Gaussian integers} are the complex numbers of the form $a+bi$ with $a$ and $b$ both integers.
\end{definition}

The sum and product of two Gaussian integers are also Gaussian integers, but the quotient need not be a Gaussian integer.

\begin{theorem}
(Gaussian Unit Theorem). The only units in the Gaussian integers are $1, -1, i,$ and $-i$. That is, these are the only Gaussian integers that have Gaussian integer multiplicative inverses.
\end{theorem}

\begin{definition}
The \textit{norm} of $x+yi$ is $N(x+yi)=x^2 + y^2$.
\end{definition}

\begin{theorem}
(Norm Multiplication Property). Let $\alpha$ and $\beta$ be any complex numbers. Then
\begin{align*}
N(\alpha\beta)=N(\alpha)N(\beta).
\end{align*}
\end{theorem}

A Gaussian integer $\alpha$ is a unit if and only if $N(\alpha)=1$.

\begin{svgraybox}
\begin{theorem}
(Gaussian Prime Theorem). The Gaussian primes can be described as follows:
\begin{enumerate}[(i)]
\item $1+i$ is a Gaussian prime.
\item Let $p$ be an ordinary prime with $p\equiv 3\bmod 4$. Then $p$ is a Gaussian prime.
\item Let $p$ be an ordinary prime with $p\equiv 1\bmod 4$ and write $p$ as a sum of two squares $p=u^2+v^2$. Then $u+vi$ is a Gaussian prime.
\end{enumerate}
Every Gaussian prime is equal to a unit ($\pm1$ or $\pm i$) multiplied by a Gaussian prime of the form (i),(ii) or (iii).
\end{theorem}
\end{svgraybox}

\begin{lemma}
(Gaussian Divisibility Lemma). Let $\alpha = a+bi$ be a Gaussian integer.
\begin{enumerate}[(a)]
\item If 2 divides $N(\alpha)$, then $1+i$ divides $\alpha$.
\item Let $pi=p$ be a category (ii) prime, and suppose that $p$ divides $N(\alpha)$ as ordinary integers. Then $pi$ divides $\alpha$ as Gaussian integers.
\item Let $pi=u+vi$ be a Gaussian prime in category (iii), and let $\overline{\pi} = u-vi$. (This is a natural notation, since $\overline{\pi}$ is indeed the complex conjugate of the complex number $\pi$.) Suppose that $N(\pi)=p$ divides $N(\alpha)$ as ordinary integers. Then at least one of $\pi$ and $\overline{\pi}$ divides $\alpha$ as Gaussian integerse.
\end{enumerate}
\end{lemma}

\section*{Chapter 36: The Gaussian Integers and Unique Factorization}

\begin{definition}
We say that $x+yi$ is \textit{normalized} if $x >0$ and $y\geqslant0$.
\end{definition}

\begin{theorem}
(Unique Factorization of Gaussian Integers). Every Gaussian integer $\alpha \not = 0$ can be factored into a unit $u$ multiplied by a product of normalized Gaussian primes
\begin{align*}
\alpha = u\pi_1\pi_2\cdots\pi_r
\end{align*}
in exactly one way.
\end{theorem}

\begin{theorem}
(Gaussian Integer Division with Remainder). Let $\alpha$ and $\beta$ be Gaussian integers with $\beta \not = 0$. Then there are Gaussian integers $\gamma$ and $\rho$ such that
\begin{align*}
\alpha = \beta\gamma+\rho \text{ and } N(\rho)< N(\beta).
\end{align*}
\end{theorem}

\begin{theorem}
(Gaussian Integer Common Divisor Property). Let $\alpha$ and $\beta$ be Gaussian integers, and let $S$ be the collection of Gaussian integers
\begin{align*}
A\alpha +B\beta, \text{ where $A$ and $B$ are any Gaussian integers.}
\end{align*}
Among all Gaussian integers in $S$, choose an element
\begin{align*}
g = a\alpha + b\beta
\end{align*}
having the smallest nonzero norm. In other words,
\begin{align*}
0 < N(g) \leqslant N(A\alpha+B\beta) \text{ for any Gaussian integers $A$ and $B$ with $A\alpha+B\beta \not = 0$}.
\end{align*}
Then $g$ divides both $\alpha$ and $\beta$.
\end{theorem}

\begin{theorem}
(Gaussian Prime Divisibility Property). Let $\pi$ be a Gaussian prime, let$\alpha$ and $\beta$ be Gaussian integers, and suppose that $pi$ divides the product $\alpha\beta$. Then either $pi$ divides $\alpha$ or $pi$ divides $\beta$ (or both).
More generally, if $pi$ divides a product $\alpha_1\alpha_2\cdots\alpha_n$ of Gaussian integers, then it divides at least one of the factors $\alpha_1, \alpha_2, \ldots, \alpha_n$.
\end{theorem}

\begin{theorem}
(Sum of Two Squares Theorem (Legendre)). For a given positive integer $N$, let\\
$D_1 =$ (the number of positive integers $d$ dividing $N$ that satisfying $d\equiv 1\bmod 4$),\\
$D_3 =$ (the number of positive integers $d$ dividing $N$ that satisfying $d\equiv 3\bmod 4$).\\
Then $N$ can be written as a sum of two squares in exactly $R(N)=4(D_1-D_3)$ ways.
\end{theorem}

\begin{theorem}
(Difference of $D_1-D_3$ Theorem). Factor the integer $N$ into a product of ordinary primes as
\begin{align*}
N = 2^t p_1^{e_1}p_2^{e_2}\cdots p_r^{e_r} \cdot q_1^{f_1}q_2^{f_2}\cdots q_s^{f_s}.
\end{align*}
where $p_i$'s are 1 mod 4 primes, $q_j$'s are 3 mod 4 primes.
Let \\
$D_1 =$ (the number of positive integers $d$ dividing $N$ that satisfying $d\equiv 1\bmod 4$),\\
$D_3 =$ (the number of positive integers $d$ dividing $N$ that satisfying $d\equiv 3\bmod 4$).\\
Then the difference $D_1-D_3$ is given by the rule
\begin{align*}
D_1-D_3= \begin{cases} (e_1+1)(e_2+1) \cdots (e_r+1) &\mbox{if $f_1, \ldots, f_s$ are all even,} \\
0 & \mbox{if any of $f_1, \ldots, f_s$ is odd.} \end{cases}
\end{align*}
\end{theorem}

\end{document}